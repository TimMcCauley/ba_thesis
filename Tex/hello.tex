\documentclass[10pt,a4paper]{article}
\usepackage[utf8]{inputenc}
\usepackage[german]{babel}
\usepackage[T1]{fontenc}
\usepackage{amsmath}
\usepackage{amsfonts}
\usepackage{amssymb}
\usepackage{color}
\author{Amandus Stefan Butzer}
\title{Erstellung eines Routing-Profils auf Basis von OSM / Öffentlichen Daten für Feuerwehrfahrzeuge}
\date{\today}

\begin{document}

\maketitle


\section{Einleitung}

Hello this is my batchelor Thesis

\subsection{Motivation}

Blahblah \ldots

\section{Theoretische Grundlagen}

\subsection{Graphen Erstellung}

\subsection{Routing}

\subsection{Isochronen Berechnung}
A footnote looks like this\footnote{Hello this is Footnote}

\section{Generierung des Routing-Profils}

\subsection{Informations Erhebung}
Fragebogen für Feuerwehr Lützelburg\footnote{Lützelburg ist eine stadt in Bayern}

\subsection{Limitierende Faktoren}

\subsection{Erweiternde Faktoren}

\section{Ergebnisse}
\paragraph{
Vergleiche zwischen Firetruck - Emergency Vehicle - Car - Heavy Vehicle
}
\paragraph{
\color{red}
Hier würde ich ein paar räumliche Beispiele aussuchen und exemplarisch zeigen (Routing und Isochronen), welche Änderungen das Profil mit sich bringt, einerseits innerstädtisch, andererseits auch außerhalb der Stadt. Denn Änderungen als solches ist bisschen schwierig zu definieren. Gerne die Jungs aus Lützelburg fragen, welches Gebiet mit den bereits vorhandenen Profilen wirklich schlechte Ergebnisse bringt und jetzt mit Emergency weitaus realitischere!
}

\section{Fazit}

tolles teil

\section{Future Work/Ausblick}
\begin{itemize}
\item Suche nach Löschwasser quellen um den Zielpunkt (osm tag emergency=fire_hydrant)
\item Beschleunigung
\item rush hour / tag & nacht unterscheidung (nachts weniger los auf straßen/ Fußgängerzonen ...)
\end{itemize}


\end{document}
